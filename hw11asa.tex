\documentclass[a4paper,11pt]{article}

\usepackage{geometry}
\usepackage{listings}
\usepackage{mathrsfs}
\usepackage[font={small},labelfont={sf,bf}]{caption}
\usepackage{color}
\usepackage{amssymb}
\usepackage[utf8]{inputenc}
\usepackage{afterpage}
\usepackage[T1]{fontenc}
\usepackage{amsfonts}
\usepackage{mathtools}
\usepackage{amsmath}
\usepackage{verbatim}
\usepackage{bm}
\usepackage{bbm}
\usepackage{enumerate}
\usepackage{amsthm}
\usepackage{stmaryrd}

\geometry{a4paper,top=3cm,bottom=3cm,left=2cm,right=2cm,heightrounded, bindingoffset=5mm}

\theoremstyle{definition}
\newtheorem{definition}{Definition}[section]
\theoremstyle{plain}
\newtheorem{theo}[definition]{Theorem}
\newtheorem{prop}[definition]{Proposition}
\newtheorem{lemma}[definition]{Lemma}
\newtheorem{cor}[definition]{Corollary}
\newtheorem{ex}[definition]{Example}
\theoremstyle{remark}
\newtheorem{rem}[definition]{Remark}
\newtheorem{rem*}[definition]{}

\newcommand*\mcup{\mathbin{\mathpalette\mcupinn\relax}}
\newcommand*\mcupinn[2]{\vcenter{\hbox{$\mathsurround=0pt
  \ifx\displaystyle#1\textstyle\else#1\fi\bigcup$}}}
\newcommand*\mcap{\mathbin{\mathpalette\mcapinn\relax}}
\newcommand*\mcapinn[2]{\vcenter{\hbox{$\mathsurround=0pt
  \ifx\displaystyle#1\textstyle\else#1\fi\bigcap$}}}
\DeclarePairedDelimiter{\abs}{\lvert}{\rvert}
\DeclarePairedDelimiter{\norm}{\lVert}{\rVert}
\DeclarePairedDelimiter{\parr}{(}{)}
\DeclarePairedDelimiter{\parq}{[}{]}
\DeclarePairedDelimiter{\parqq}{\llbracket}{\rrbracket}
\DeclarePairedDelimiter{\bra}{\lbrace}{\rbrace}
\DeclarePairedDelimiter{\ceil}{\lceil}{\rceil}
\DeclarePairedDelimiter{\prodscal}{\langle}{\rangle}
\DeclarePairedDelimiter{\floor}{\lfloor}{\rfloor}
\DeclareMathOperator*{\argmin}{arg\,min}
\DeclareMathOperator*{\argmax}{arg\,max}
\DeclareMathOperator*{\expval}{\mathbb{E}}
\DeclareMathOperator*{\varval}{\mathrm{Var}}
\DeclareMathOperator*{\covval}{\mathrm{Cov}}

\begin{document}

\title{Applied Stochastic Analysis \\ Homework assignment 11}
\author{Luca Venturi}
\maketitle

\section*{Exercise 1}

\paragraph*{(a)}

The corresponding Fokker-Plank equation is
\begin{align*}
\rho_t & = -\partial_x(\lambda x\rho) + \partial^2_{xx}\parr*{\frac{1}{2}\sigma^2x^2\rho} = (\sigma^2-\lambda)\rho + (2\sigma^2 -\lambda)x\rho_x + \frac{1}{2}\sigma^2x^2\rho_{xx}.
\end{align*}

\paragraph*{(b)}

If $\rho=\rho(x)$ is a stationary distribution, then it must satisfy
\begin{equation}
0 = \mathcal{L}^*\rho = (\sigma^2-\lambda)\rho + (2\sigma^2 -\lambda)x\rho' + \frac{1}{2}\sigma^2x^2\rho''.\label{ex1:eq:stat_dis}
\end{equation}
This is a second order ODE, so it admits a 2 dimensional family of solutions. In particular we look for solutions of the form $\rho(x)=x^\alpha$, $\alpha\in\mathbb{R}$. In this case (\ref{ex1:eq:stat_dis}) becomes
$$
0 = x^\alpha\parq*{(\sigma^2-\lambda) + \alpha(2\sigma^2 -\lambda) + \frac{1}{2}\sigma^2\alpha(\alpha-1)}
$$
which implies
$$
\frac{1}{2}\sigma^2\alpha^2 +\parr*{\frac{3}{2}\sigma^2-\lambda}\alpha + (\sigma^2-\lambda) = 0,
$$
i.e.
\begin{align*}
\alpha & = \frac{1}{\sigma^2}\parq*{\parr*{\lambda-\frac{3}{2}\sigma^2}\pm\sqrt{\parr*{\frac{3}{2}\sigma^2-\lambda}^2-2\sigma^2(\sigma^2-\lambda)}} \\ & = \frac{1}{\sigma^2}\parq*{\parr*{\lambda-\frac{3}{2}\sigma^2}\pm\sqrt{\parr*{\frac{1}{2}\sigma^2-\lambda}^2}} =\begin{cases}-1 \\ \frac{2(\lambda-\sigma^2)}{\sigma^2}\end{cases}.
\end{align*}
Therefore the stationary distribution must be of the form 
$$
\rho_s(x) = \frac{c_1}{x} + c_2 x^{\frac{2(\lambda-\sigma^2)}{\sigma^2}}
$$
for some constants $c_1,c_2\in\mathbb{R}$ (a part from the case $\sigma^2=2\lambda$; in this case it must have the form $\rho_s(x) = \frac{c_1}{x} + c_2\frac{\log x}{x}$). In any case no one of these functions is non negative integrable; therefore they can not be distributions. This means no stationary distribution exists.

\paragraph*{(c)}

The $n$-th moment $M_n$ must satisfy the PDE given by the backward Kolmogorov equation:
$$
\partial_t M_n(x,t) = \mathcal{L} M_n (x,t) = \lambda x\partial_{x}M_n(x,t) + \frac{1}{2}\sigma^2x^2\partial^2_{xx}M_n(x,t),
$$
along with the initial conditions $M_n(x,0) = x^n$.

\section*{Exercise 2}

\paragraph*{(a)}

The boundary conditions for the operator $\mathcal{L}$ acting on $f=f(x,t)$ are given by 
$$
(\mathbf{j}\cdot\mathbf{n})f  + \rho(a \cdot \nabla f)\cdot \mathbf{n} = 0 \quad\text{at } x=0 \quad\Leftrightarrow\quad j(0,t)f(0,t) + \frac{1}{2}\rho(0,t)f'(0,t)=0.
$$
Using the boundary condition for $\mathcal{L}$, the above equation becomes 
$$
\rho(0,t)\parr*{\kappa f(0,t)+ \frac{1}{2}f'(0,t)} = 0.
$$

\paragraph*{(b)}

We have that
\begin{align*}
\dot{P}_{tot}(t) & = \int_0^\infty\partial_t\rho(x,t)\,dx = \int_0^\infty\mathcal{L}^*\rho(x,t)\,dx = \alpha\int_0^\infty\partial_x(x\rho(x,t))\,dx + \frac{1}{2}\int_0^\infty\partial_{xx}\rho(x,t)\,dx \\  & = \alpha\parq*{x\rho(x,t)}_0^\infty + \frac{1}{2}\parq*{\partial_x\rho(x,t)}_0^\infty = -\frac{1}{2}\partial_x\rho(0,t),
\end{align*}
where we assumed that $\lim_{x\to+\infty}(x\rho(x,t))=0$ and $\lim_{x\to+\infty}\partial_x\rho(x,t)=0$.
Using the boundary condition
$$
\kappa \rho(0,t) = -j(0,t) = -\frac{1}{2}\partial_x\rho(0,t),
$$
we get that $\dot{P}_{tot}(t) = \kappa \rho(0,t)$. Thus the total probability is only conserved if $\rho(0,t) \equiv 0$.

\section*{Exercise 3}

\paragraph*{(a)}

The process $X_t\in[0,L]$ should satisfy the two SDEs:
$$
\begin{aligned}
dX_t & = -v\,dt + \sigma\,dW_t \quad && \text{if }X_t \in [0,d),
\\ dX_t & = \sigma\,dW_t  && \text{if }X_t \in [d,L].
\end{aligned}
$$

\paragraph*{(b)}

The backward equation for $\rho = \rho (x,t)$, $t\geq0, x\in[0,L]$, is 
$$
\rho_t = \mathcal{L}^*\rho =  -v\mathbbm{1}_{\bra{x\in[0,d]}}\rho_x + \frac{1}{2}\sigma^2\rho_{xx},
$$
along with the boundary conditions
$$
0 = \partial_x\rho(L,t) = v\rho(0,t) - \frac{1}{2}\sigma^2\partial_x\rho(0,t)
$$
and initial condition $\rho(0,t)=\rho_0(t)$.
The forward equation for $u = u(x,t) = \expval^x f(X_t)$, $t\geq0, x\in[0,L]$, is 
$$
u_t = \mathcal{L}u =  v\mathbbm{1}_{\bra{x\in[0,d]}}u_x + \frac{1}{2}\sigma^2u_{xx},
$$
along with the boundary conditions
$0=\partial_xu(L,t)=\partial_xu(0,t)$ and initial condition $u(x,0)=f(x)$.

\paragraph*{(c)}

The stationary distribution $\rho_s(x)=\rho^{(1)}(x)+\rho^{(2)}(x)$, where we denote $\rho^{(1)} = \rho_s|_{x\in[0,d]}$ and $\rho^{(2)} = \rho_s|_{x\in[d,L]}$. Then
$\rho^{(1)}$ satisfies 
$$
0 = v\rho^{(1)}_x + \frac{1}{2}\sigma^2\rho^{(1)}_{xx} \quad\Rightarrow\quad \rho^{(1)}(x) = A_1e^{2vx/\sigma^2} - \frac{1}{v}B_1
$$
and $\rho^{(2)}$ satisfies
$$
0 = \frac{1}{2}\sigma^2\rho^{(2)}_{xx} \quad\Rightarrow\quad \rho^{(2)}(x) = A_2x+B_2.
$$
If we impose the boundary (and continuity) conditions we get
\begin{align*}
v\rho^{(1)}(0)-\frac{1}{2}\sigma^2\partial_x\rho^{(1)}(0) = 0 & \quad\Rightarrow\quad vA_1 - B_1 -\frac{1}{2}\sigma^2A_1\frac{2v}{\sigma^2} = 0 \quad\Rightarrow\quad B_1 = 0,\\
\partial_x\rho(L) = 0 & \quad\Rightarrow\quad A_2 = 0,\\
\rho^{(1)}(d) = \rho^{(2)}(d) & \quad\Rightarrow\quad B_2 = A_1e^{2vd/\sigma^2}.
\end{align*}
Also, we can find $A_1$ by imposing $\int_0^L\rho_s(x)\,dx= 1$, i.e.
$$
A_1 = \parq*{\frac{\sigma^2}{2v}\parr*{e^{2vd/\sigma^2}-1} + (L-d)e^{2vd/\sigma^2}}^{-1}.
$$
Therefore the stationary distribution is given by
$$
\rho_s(x) = A_1\mathrm{exp}\bra*{\frac{2v}{\sigma^2}\parr*{x\mathbbm{1}_{\bra{x\in[0,d]}}+
d\mathbbm{1}_{\bra{x\in(d,L]}}}}.
$$

\section*{Exercise 4}

\paragraph*{(a)}

Be $V_t = \dot{X_t}$. Then we can write the Langevin equation as a first order system:
$$
\begin{cases} dX_t = V_t\,dt \\ dV_t = -\frac{1}{m}(\gamma V_t + \nabla U(X_t))\,dt + \frac{1}{m}\sqrt{2}\sigma\,dW_t \end{cases}
$$
i.e.
$$
d\mathbf{X}_t = \mathbf{b}(\mathbf{X}_t)\,dt + \Sigma\,d\mathbf{W}_t,
$$
where
$$
\mathbf{X}_t = \left(\begin{matrix}
X_t \\ V_t
\end{matrix}\right), \quad \mathbf{b}(\mathbf{X}_t) = \left(\begin{matrix}
V_t \\ -\frac{1}{m}(\gamma V_t + \nabla U(X_t))
\end{matrix}\right) \quad \text{and} \quad \Sigma = \left(\begin{matrix}
0 & 0 \\ \frac{1}{m}\sqrt{2}\sigma & 0
\end{matrix}\right).
$$

\paragraph*{(b)}

The corresponding Fokker-Plank equation is
\begin{equation}
\rho_t = \mathcal{L}^*_{(x,v)} \rho = -\nabla_{(x,v)} \cdot (\mathbf{b}\rho) + \nabla^2_{(x,v)}:(A\rho)\label{ex4:1d_fp}
\end{equation}
where
$$
A = \frac{1}{2}\Sigma\Sigma^T = \left(\begin{matrix}
0 & 0 \\ 0 & \frac{\sigma^2}{m^2}
\end{matrix}\right).
$$
Equation (\ref{ex4:1d_fp}) can be written more explicitly as
\begin{equation}
\rho_t = -v\rho_x + \frac{1}{m}[\rho_v(\gamma v + \nabla U(x))+ \gamma\rho] + \frac{\sigma^2}{m^2}\rho_{vv}
\label{ex4:1d_fp:2}
\end{equation}

\paragraph*{(c)}

If $\rho = \rho_s(x,v) = Z^{-1}e^{-\beta H(x,v)}$, we have
\begin{align*}
-v\rho_x & = -v(-\beta H_x)\rho = \beta v \nabla U(x)\rho,
\\ \frac{1}{m}\rho_v(\gamma v + \nabla U(x)) & = -\frac{\beta}{m}H_v(\gamma v + \nabla U(x))\rho = -\beta v\nabla U(x)\rho - \beta\gamma v^2\rho,
\\ \frac{\sigma^2}{m^2}\rho_{vv} & = \frac{\sigma^2}{m^2}(\rho H_v)_v = -\beta\frac{\sigma^2}{m^2} H_{vv} +\frac{\sigma^2}{m^2}\beta^2(H_v)^2 = \frac{\gamma}{m}\rho + \beta\gamma v^2\rho.
\end{align*}
Then, using the above equations, (\ref{ex4:1d_fp:2}) gives
$$
\rho_t = \mathcal{L}^*_{(x,v)} \rho = \beta v \nabla U(x)\rho -\beta v\nabla U(x)\rho - \beta\gamma v^2\rho + \frac{\gamma}{m}\rho + \frac{\gamma}{m}\rho + \beta\gamma v^2\rho = 0.
$$

\paragraph*{(d)}

The steady-state flux is 
\begin{align*}
\mathbf{j}_s = \mathbf{b}\rho - \nabla_{(x,v)}\cdot(A\rho) = \rho\left(\begin{matrix}
v \\ -\frac{1}{m}(\gamma v + \nabla U(x))
\end{matrix}\right) - \rho_v\left(\begin{matrix}
0 \\ \frac{\sigma^2}{m^2}
\end{matrix}\right).
\end{align*}
From this expression we see that $\mathbf{j}_s$ is not zero in general and neither if $\rho = \rho_s(x,v) = Z^{-1}e^{-\beta H(x,v)}$.

\paragraph*{(e)} 

Be $\mathbf{X}_t\in\mathbb{R}^n$, $\mathbf{V}_t = \dot{\mathbf{X}_t}$ and $\mathbf{Y}_t = (\mathbf{X}_t,\mathbf{V}_t)$. Then $\mathbf{Y}_t$ satisfies the SDEs system
$$
d\mathbf{Y}_t = \mathbf{b}(\mathbf{Y}_t)\, dt + \Sigma(\mathbf{Y}_t)\,d\mathbf{W}_t,
$$
where
$$
\mathbf{b}(\mathbf{Y}_t) = \left(\begin{matrix}
\mathbf{V}_t \\ -\frac{1}{m}(\gamma(\mathbf{X}_t) \mathbf{V}_t + \nabla U(\mathbf{X}_t))
\end{matrix}\right) \quad \text{and} \quad \Sigma(\mathbf{Y}_t) = \left(\begin{matrix} \mathbf{0} & \mathbf{0} \\ \frac{1}{m}\sqrt{2}\sigma(\mathbf{X}_t) & \mathbf{0} \end{matrix}\right).
$$
The corresponding Fokker-Plank equation is
\begin{equation}
\rho_t = \mathcal{L}^*_{(x,v)} \rho = -\nabla_{(x,v)} \cdot (\mathbf{b}\rho) + \nabla^2_{(x,v)}:(A\rho)\label{ex4:nd_fp}
\end{equation}
where
$$
A = A(\mathbf{x}) = \frac{1}{2}\Sigma(\mathbf{x})\Sigma(\mathbf{x})^T = \left(\begin{matrix}
0 & 0 \\ 0 & \frac{\beta}{m^2}\gamma(\mathbf{x})
\end{matrix}\right).
$$

\end{document}