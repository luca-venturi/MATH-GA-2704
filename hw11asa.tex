\documentclass[a4paper,11pt]{article}

\usepackage{geometry}
\usepackage{listings}
\usepackage{mathrsfs}
\usepackage[font={small},labelfont={sf,bf}]{caption}
\usepackage{color}
\usepackage{amssymb}
\usepackage[utf8]{inputenc}
\usepackage{afterpage}
\usepackage[T1]{fontenc}
\usepackage{amsfonts}
\usepackage{mathtools}
\usepackage{amsmath}
\usepackage{verbatim}
\usepackage{bm}
\usepackage{bbm}
\usepackage{enumerate}
\usepackage{amsthm}
\usepackage{stmaryrd}

\geometry{a4paper,top=3cm,bottom=3cm,left=2cm,right=2cm,heightrounded, bindingoffset=5mm}

\theoremstyle{definition}
\newtheorem{definition}{Definition}[section]
\theoremstyle{plain}
\newtheorem{theo}[definition]{Theorem}
\newtheorem{prop}[definition]{Proposition}
\newtheorem{lemma}[definition]{Lemma}
\newtheorem{cor}[definition]{Corollary}
\newtheorem{ex}[definition]{Example}
\theoremstyle{remark}
\newtheorem{rem}[definition]{Remark}
\newtheorem{rem*}[definition]{}

\newcommand*\mcup{\mathbin{\mathpalette\mcupinn\relax}}
\newcommand*\mcupinn[2]{\vcenter{\hbox{$\mathsurround=0pt
  \ifx\displaystyle#1\textstyle\else#1\fi\bigcup$}}}
\newcommand*\mcap{\mathbin{\mathpalette\mcapinn\relax}}
\newcommand*\mcapinn[2]{\vcenter{\hbox{$\mathsurround=0pt
  \ifx\displaystyle#1\textstyle\else#1\fi\bigcap$}}}
\DeclarePairedDelimiter{\abs}{\lvert}{\rvert}
\DeclarePairedDelimiter{\norm}{\lVert}{\rVert}
\DeclarePairedDelimiter{\parr}{(}{)}
\DeclarePairedDelimiter{\parq}{[}{]}
\DeclarePairedDelimiter{\parqq}{\llbracket}{\rrbracket}
\DeclarePairedDelimiter{\bra}{\lbrace}{\rbrace}
\DeclarePairedDelimiter{\ceil}{\lceil}{\rceil}
\DeclarePairedDelimiter{\prodscal}{\langle}{\rangle}
\DeclarePairedDelimiter{\floor}{\lfloor}{\rfloor}
\DeclareMathOperator*{\argmin}{arg\,min}
\DeclareMathOperator*{\argmax}{arg\,max}
\DeclareMathOperator*{\expval}{\mathbb{E}}
\DeclareMathOperator*{\varval}{\mathrm{Var}}
\DeclareMathOperator*{\covval}{\mathrm{Cov}}

\begin{document}

\title{Applied Stochastic Analysis \\ Homework assignment 11}
\author{Luca Venturi}
\maketitle

\section*{Exercise 1}

\paragraph*{(a)}

The corresponding Fokker-Plank equations are
\begin{align*}
\rho_t & = -\partial_x(\lambda x\rho) + \partial^2_{xx}\parr*{\frac{1}{2}\sigma^2x^2\rho} \\ & = (\sigma^2-\lambda)\rho + (2\sigma^2 -\lambda)x\rho_x + \frac{1}{2}\sigma^2x^2\rho_{xx}.
\end{align*}

\section*{Exercise 3}

\paragraph*{(a)}

$X_t$ should satisfy the SDE
$$
dX_t = -v\mathbbm{1}_{\bra{X_t\in[0,d]}}\,dt + \sigma\,dW_t, \qquad X_t \in [0,L].
$$

\paragraph*{(b)}

$X_t$ should satisfy the SDE
$$
dX_t = -v\mathbbm{1}_{\bra{X_t\in[0,d]}}\,dt + \sigma\,dW_t, \qquad X_t \in [0,L].
$$

\section*{Exercise 4}

\paragraph*{(a)}

Be $V_t = \dot{X_t}$. Then we can write the Langevin equation as a first order system:
$$
\begin{cases} dX_t = V_t\,dt \\ dV_t = -\frac{1}{m}(\gamma V_t + \nabla U(X_t))\,dt + \frac{1}{m}\sqrt{2}\sigma\,dW_t \end{cases}
$$
i.e.
$$
d\mathbf{X}_t = \mathbf{b}(\mathbf{X}_t)\,dt + \Sigma\,d\mathbf{W}_t,
$$
where
$$
\mathbf{X}_t = \left(\begin{matrix}
X_t \\ V_t
\end{matrix}\right), \quad \mathbf{b}(\mathbf{X}_t) = \left(\begin{matrix}
V_t \\ -\frac{1}{m}(\gamma V_t + \nabla U(X_t))
\end{matrix}\right) \quad \text{and} \quad \Sigma = \left(\begin{matrix}
0 & 0 \\ \frac{1}{m}\sqrt{2}\sigma & 0
\end{matrix}\right).
$$

\paragraph*{(b)}

The corresponding Fokker-Plank equation is
\begin{equation}
\rho_t = \mathcal{L}^*_{(x,v)} \rho = -\nabla_{(x,v)} \cdot (\mathbf{b}\rho) + \nabla^2_{(x,v)}:(A\rho)\label{ex4:1d_fp}
\end{equation}
where
$$
A = \frac{1}{2}\Sigma\Sigma^T = \left(\begin{matrix}
0 & 0 \\ 0 & \frac{\sigma^2}{m^2}
\end{matrix}\right).
$$
Equation (\ref{ex4:1d_fp}) can be written more explicitly as
\begin{equation}
\rho_t = -v\rho_x + \frac{1}{m}[\rho_v(\gamma v + \nabla U(x))+ \gamma\rho] + \frac{\sigma^2}{m^2}\rho_{vv}
\label{ex4:1d_fp:2}
\end{equation}

\paragraph*{(c)}

If $\rho = \rho_s(x,v) = Z^{-1}e^{-\beta H(x,v)}$, we have
\begin{align*}
-v\rho_x & = -v(-\beta H_x)\rho = \beta v \nabla U(x)\rho,
\\ \frac{1}{m}\rho_v(\gamma v + \nabla U(x)) & = -\frac{\beta}{m}H_v(\gamma v + \nabla U(x))\rho = -\beta v\nabla U(x)\rho - \beta\gamma v^2\rho,
\\ \frac{\sigma^2}{m^2}\rho_{vv} & = \frac{\sigma^2}{m^2}(\rho H_v)_v = -\beta\frac{\sigma^2}{m^2} H_{vv} +\frac{\sigma^2}{m^2}\beta^2(H_v)^2 = \frac{\gamma}{m}\rho + \beta\gamma v^2\rho.
\end{align*}
Then, using the above equations, (\ref{ex4:1d_fp:2}) gives
$$
\rho_t = \mathcal{L}^*_{(x,v)} \rho = \beta v \nabla U(x)\rho -\beta v\nabla U(x)\rho - \beta\gamma v^2\rho + \frac{\gamma}{m}\rho + \frac{\gamma}{m}\rho + \beta\gamma v^2\rho = 0.
$$

\paragraph*{(d)}

The steady-state flux is 
\begin{align*}
\mathbf{j}_s = \mathbf{b}\rho - \nabla_{(x,v)}\cdot(A\rho) = \rho\left(\begin{matrix}
v \\ -\frac{1}{m}(\gamma v + \nabla U(x))
\end{matrix}\right) - \rho_v\left(\begin{matrix}
0 \\ \frac{\sigma^2}{m^2}
\end{matrix}\right).
\end{align*}
From this expression we see that $\mathbf{j}_s$ is not zero in general and neither if $\rho = \rho_s(x,v) = Z^{-1}e^{-\beta H(x,v)}$.

\paragraph*{(e)} fare

\end{document}