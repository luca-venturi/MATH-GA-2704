\documentclass[a4paper,11pt]{article}

\usepackage{geometry}
\usepackage{listings}
\usepackage{mathrsfs}
\usepackage[font={small},labelfont={sf,bf}]{caption}
\usepackage{color}
\usepackage{amssymb}
\usepackage[utf8]{inputenc}
\usepackage{afterpage}
\usepackage[T1]{fontenc}
\usepackage{amsfonts}
\usepackage{mathtools}
\usepackage{amsmath}
\usepackage{verbatim}
\usepackage{bm}
\usepackage{bbm}
\usepackage{enumerate}
\usepackage{amsthm}
\usepackage{stmaryrd}

\geometry{a4paper,top=3cm,bottom=3cm,left=2cm,right=2cm,heightrounded, bindingoffset=5mm}

\theoremstyle{definition}
\newtheorem{definition}{Definition}[section]
\theoremstyle{plain}
\newtheorem{theo}[definition]{Theorem}
\newtheorem{prop}[definition]{Proposition}
\newtheorem{lemma}[definition]{Lemma}
\newtheorem{cor}[definition]{Corollary}
\newtheorem{ex}[definition]{Example}
\theoremstyle{remark}
\newtheorem{rem}[definition]{Remark}
\newtheorem{rem*}[definition]{}

\newcommand*\mcup{\mathbin{\mathpalette\mcupinn\relax}}
\newcommand*\mcupinn[2]{\vcenter{\hbox{$\mathsurround=0pt
  \ifx\displaystyle#1\textstyle\else#1\fi\bigcup$}}}
\newcommand*\mcap{\mathbin{\mathpalette\mcapinn\relax}}
\newcommand*\mcapinn[2]{\vcenter{\hbox{$\mathsurround=0pt
  \ifx\displaystyle#1\textstyle\else#1\fi\bigcap$}}}
\DeclarePairedDelimiter{\abs}{\lvert}{\rvert}
\DeclarePairedDelimiter{\norm}{\lVert}{\rVert}
\DeclarePairedDelimiter{\parr}{(}{)}
\DeclarePairedDelimiter{\parq}{[}{]}
\DeclarePairedDelimiter{\parqq}{\llbracket}{\rrbracket}
\DeclarePairedDelimiter{\bra}{\lbrace}{\rbrace}
\DeclarePairedDelimiter{\ceil}{\lceil}{\rceil}
\DeclarePairedDelimiter{\prodscal}{\langle}{\rangle}
\DeclarePairedDelimiter{\floor}{\lfloor}{\rfloor}
\DeclareMathOperator*{\argmin}{arg\,min}
\DeclareMathOperator*{\argmax}{arg\,max}
\DeclareMathOperator*{\expval}{\mathbb{E}}
\DeclareMathOperator*{\varval}{\mathrm{Var}}
\DeclareMathOperator*{\covval}{\mathrm{Cov}}

\begin{document}

\title{Applied Stochastic Analysis \\ Homework assignment 8}
\author{Luca Venturi}
\maketitle

\section*{Exercise 1}

\paragraph*{(a)}

We use the same notation as in the text of the assignment. Here $\sum_j$ means $\sum_{j=0}^{n-1}$. It holds that
\begin{align*}
\sum_j\frac{1}{2}(W_{j+1}+W_j)\Delta W_j = \frac{1}{2}\sum_j(W_{j+1}^2-W_j^2) = \frac{1}{2}W(t)^2.
\end{align*}
Therefore it is straightforward that 
$$
\sum_j\frac{1}{2}(W_{j+1}+W_j)\Delta W_j \quad\xrightarrow{L^2}\quad \frac{1}{2}W(t)^2
$$
as $|\sigma|\to0$.

\paragraph*{(b)}

We consider 
\begin{align*}
\sum_j\frac{1}{2}(W_{j+1}+W_j)\Delta W_j - \sum_jW_{j+1/2}\Delta W_j & = \frac{1}{2}\sum_j(W_{j+1}-2W_{j+1/2}+W_j)\Delta W_j \\ & = \frac{1}{2}\sum_j(W_{j+1}-W_{j+1/2})^2 -\frac{1}{2}\sum_j(W_{j+1/2}-W_j)^2.
\end{align*}
Now 
\begin{align*}
& \mathbb{E}\Big[\Big(\sum_j[(W_{j+1}-W_{j+1/2})^2-(t_{j+1}-t_{j+1/2})]\Big)^2\Big] = \\ & = \sum_{i,j}\mathbb{E}[((W_{j+1}-W_{j+1/2})^2-(t_{j+1}-t_{j+1/2}))((W_{i+1}-W_{i+1/2})^2-(t_{i+1}-t_{i+1/2}))] \\ & = \sum_i \mathbb{E}[((W_{i+1}-W_{i+1/2})^2-(t_{i+1}-t_{i+1/2}))^2] \leq \sum_i \mathbb{E}[(W_{i+1}-W_{i+1/2})^4] + \sum_i(t_{i+1}-t_{i+1/2})^2 \\ & = 4 \sum_i(t_{i+1}-t_{i+1/2})^2 \leq 2 |\sigma| \sum_i(t_{i+1}-t_{i+1/2}) = |\sigma|,
\end{align*}
which implies that 
$$
\sum_j(W_{j+1}-W_{j+1/2})^2 \quad\xrightarrow{L^2}\quad \frac{t}{2}
$$
as $|\sigma|\to0$.
Similarly one can prove that
$$
\sum_j(W_{j+1/2}-W_j)^2 \quad\xrightarrow{L^2}\quad \frac{t}{2}
$$
as $|\sigma|\to0$.
Thus, it follows that
$$
\sum_j\frac{1}{2}(W_{j+1}+W_j)\Delta W_j - \sum_jW_{j+1/2}\Delta W_j \quad\xrightarrow{L^2}\quad \frac{t}{4} - \frac{t}{4} = 0
$$
as $|\sigma|\to0$, which, thanks to part (a) of the exercise, implies that
$$
\sum_jW_{j+1/2}\Delta W_j \quad\xrightarrow{L^2}\quad \frac{1}{2}W(t)^2
$$
as $|\sigma|\to0$.

\paragraph*{(c)}

It holds that
\begin{align}
\sum_j W_j^2\Delta W_j & = \sum_jW_j(W_j - W_{j+1} + W_{j+1})\Delta W_j = - \sum_jW_j(\Delta W_j)^2 + \sum_j W_j W_{j+1}\Delta W_j \notag \\ & =  - \sum_jW_j(\Delta W_j)^2 + \frac{1}{3}\sum_j(W^3_{j+1}-W^3_j) -\frac{1}{3}\sum_j(W_{j+1}-W_j)^3 \notag \\ & =  - \sum_jW_j(\Delta W_j)^2 + \frac{1}{3}W(t)^3 -\frac{1}{3}\sum_j(W_{j+1}-W_j)^3. \label{ex:1.c}
\end{align}
Now 
\begin{align*}
& \mathbb{E}\Big[\Big(\sum_jW_j((W_{j+1}-W_j)^2-(t_{j+1}-t_j))\Big)^2\Big] = \\ & = \sum_{i,j}\mathbb{E}[W_iW_j((W_{i+1}-W_i)^2-(t_{i+1}-t_i))((W_{j+1}-W_j)^2-(t_{j+1}-t_j))] \\ & = \sum_i\mathbb{E}[W_i^2((W_{i+1}-W_i)^2-(t_{i+1}-t_i))^2] = 2\sum_i t_i(t_{i+1}-t_i)^2 \leq 2|\sigma|t^2 
\end{align*}
which implies that
\begin{equation}\label{ex:1.c1}
- \sum_jW_j(\Delta W_j)^2 \quad\xrightarrow{L^2}\quad -\int_0^t W(s)\,ds
\end{equation}
as $|\sigma|\to0$. Also
\begin{align*}
\mathbb{E}\Big[\Big(\sum_j(W_{j+1}-W_j)^3\Big)^2\Big] & = \sum_{i,j}\mathbb{E}[(W_{i+1}-W_i)^3(W_{j+1}-W_j)^3] = \sum_{j}\mathbb{E}[(W_{j+1}-W_j)^6] \\ & = 15\sum_j(t_{j+1}-t_j)^3 \leq 15|\sigma|^2t \quad\rightarrow\quad 0 
\end{align*}
as $|\sigma|\to0$, which implies that
\begin{equation}\label{ex:1.c2}
-\frac{1}{3}\sum_j(W_{j+1}-W_j)^3 \quad\xrightarrow{L^2}\quad 0
\end{equation}
as $|\sigma|\to0$.
Therefore (\ref{ex:1.c}), (\ref{ex:1.c1}) and (\ref{ex:1.c2}) imply that
$$
\sum_j W_j^2\Delta W_j \quad\xrightarrow{L^2}\quad \frac{1}{3}W(t)^3 -\int_0^t W(s)\,ds
$$
as $|\sigma|\to0$.

\section*{Exercise 2}

\paragraph*{(a)} 

Let $Y_t = \frac{1}{3}W_t^3$. By It\^o formula we have that
$$
dY_t = W_t^2\,dW_t + W_t\,(dW_t)^2 = W_t^2\,dW_t + W_t\,dt.
$$
Therefore it holds that
$$
\int_0^t W_s^2\,dW_s = \int_0^t\,d(\frac{1}{3}W_s^3) - \int_0^tW_s\,ds = \frac{1}{3}W_t^3 - \int_0^tW_s\,ds.
$$
Note that this result is in accordance with what found in Exercise 1.(c).

\paragraph*{(b)} 

By It\^o isometry and Fubini theorem, it holds that
$$
\mathbb{E}\Big[\Big(\int_0^t W_s^2\,dW_s\Big)^2\Big] = \mathbb{E}\Big[\int_0^tW_s^4\,ds\Big] = \int_0^t\mathbb{E}[W_s^4]\,ds = \int_0^t 3s^2\,ds = t^3. 
$$

\section*{Exercise 3}

By applying It\^o formula we find:

\paragraph*{(a)}

If $Y_t = W_t/(1+t)$ then
$$
dY_t = -\frac{1}{(1+t)^2}W_t\,dt + \frac{1}{1+t}\,dW_t.
$$

\paragraph*{(b)}

If $Y_t = \sin W_t$ then
$$
dY_t = \cos W_t\,dW_t-\frac{1}{2}\sin W_t (dW_t)^2 = -\frac{1}{2}\sin W_t\,dt + \cos W_t\,dW_t.
$$

\paragraph*{(c)}

If $X_t = a\cos W_t$, $Y_t = b\sin W_t$ ($ab\neq 0$) then
\begin{align*}
dX_t & = -a\sin W_t\,dW_t-\frac{a}{2}\cos W_t\,(dW_t)^2 = -\frac{1}{2}X_t\,dt -\frac{a}{b}Y_t\,dW_t, \\
dY_t & = b\cos W_t\,dW_t -\frac{b}{2}\sin W_t\,(dW_t)^2 = -\frac{1}{2}Y_t\,dt + \frac{b}{a}X_t\,dW_t. 
\end{align*}
We can write this as
$$
d\left(\begin{matrix}
X_t \\ Y_t
\end{matrix}\right) = -\frac{1}{2}\left(\begin{matrix}
X_t \\ Y_t
\end{matrix}\right)\,dt + \left(\begin{matrix}
0 & -a/b \\ b/a & 0
\end{matrix}\right)\left(\begin{matrix}
X_t \\ Y_t
\end{matrix}\right)\,dW_t.
$$

\section*{Exercise 4}

Let's consider $Z_t=f(X_t,Y_t) = X_tY_t$. Since 
$$
\nabla f(x,y) = (y,x) \qquad\text{and}\qquad \nabla^2f(x,y) = \left(\begin{matrix}
0 & 1 \\ 1 & 0
\end{matrix}\right) 
$$
then 
$$
\nabla f(X_t,Y_t) \cdot d(X_t,Y_t) = Y_tdX_t + X_tdY_t \qquad\text{and}\qquad d(X_t,Y_t)^T\,\nabla^2f(X_t,Y_t)\,d(X_t,Y_t) = 2dX_tdY_t. 
$$
It follows from the multi-dimensional It\^o formula that
$$
dZ_t = \nabla f(X_t,Y_t) \cdot d(X_t,Y_t) + \frac{1}{2}d(X_t,Y_t)^T\,\nabla^2f(X_t,Y_t)\,d(X_t,Y_t) = Y_tdX_t + X_tdY_t + dX_tdY_t.
$$

\section*{Exercise 5}

We have that
$$
(dX_t)^T\,\nabla^2f(X_t)\,dX_t = \sum_{i,j}dX_t^i\,\partial_{ij}f(X_t)\,dX_t^j.
$$
Now 
$$
dX_t^i\,dX_t^j = \Big(b^i\,dt + \sum_k\sigma^{ik}\,dW^k_t\Big)\,\Big(b^j\,dt + \sum_k\sigma^{jl}\,dW^l_t\Big) = \Big(\sum_k\sigma^{ik}\sigma^{jk}\Big)\,dt = (\sigma\,\sigma^T)_{ij}\,dt,
$$
thanks to the formal rules $dt\,dt = dt\,dW_t^i = dW_t^i\,dW_t^j = 0$ if $i\neq j$ and $(dW_t^i)^2 = dt$.
Therefore
$$
(dX_t)^T\,\nabla^2f(X_t)\,dX_t = \sum_{i,j} \partial_{ij}f(X_t)\,(\sigma\,\sigma^T)_{ij}\,dt = (\sigma\,\sigma^T\;:\;\nabla^2f)\,dt.
$$

\section*{Exercise 6}

Let $R_t = f(B_t)$ where $f(\mathbf{x})=|\mathbf{x}|$, $\mathbf{x}=(x_1,\dots,x_n)$. Since $\nabla f(\mathbf{x}) = \mathbf{x}/|\mathbf{x}|$ it follows that
\begin{equation}\label{ex:6}
\nabla f(B_t) \cdot dB_t = \frac{1}{|B_t|}B_t\cdot dB_t = \frac{1}{R_t}\sum_iB_i\,dB_i.
\end{equation}
Moreover $\partial_{ii}f(\mathbf{x}) = (|\mathbf{x}|^2-x^2_i)/|\mathbf{x}|^3$, so $\sum_i\partial_{ii}f(\mathbf{x}) = (n-1)/|\mathbf{x}|$. Hence
\begin{equation}\label{ex:6.1}
(dB_t)^T\,\nabla^2 f(B_t)\,dB_t = (\sigma\,\sigma^T\;:\;\nabla^2f)\,dt =\mathrm{tr}(\nabla^2 f)\,dt = \frac{n-1}{|B_t|}\,dt = \frac{n-1}{R_t}\,dt,
\end{equation}
thanks to what we proved in Exercise 5 and to the fact that $\sigma = I$ in this case.
Therefore, by (\ref{ex:6}), (\ref{ex:6.1}) and the multi-dimensional It\^o formula we get:
$$
dR_t = \nabla f(B_t) \cdot dB_t + \frac{1}{2}(dB_t)^T\,\nabla^2 f(B_t)\,dB_t = \frac{1}{R_t}\sum_iB_i\,dB_i + \frac{1}{2}\frac{n-1}{R_t}\,dt.
$$

\end{document}