\documentclass[a4paper,11pt]{article}

\usepackage{geometry}
\usepackage{listings}
\usepackage{mathrsfs}
\usepackage[font={small},labelfont={sf,bf}]{caption}
\usepackage{color}
\usepackage{amssymb}
\usepackage[utf8]{inputenc}
\usepackage{afterpage}
\usepackage[T1]{fontenc}
\usepackage{amsfonts}
\usepackage{mathtools}
\usepackage{amsmath}
\usepackage{verbatim}
\usepackage{bm}
\usepackage{bbm}
\usepackage{enumerate}
\usepackage{amsthm}
\usepackage{stmaryrd}

\geometry{a4paper,top=3cm,bottom=3cm,left=2cm,right=2cm,heightrounded, bindingoffset=5mm}

\theoremstyle{definition}
\newtheorem{definition}{Definition}[section]
\theoremstyle{plain}
\newtheorem{theo}[definition]{Theorem}
\newtheorem{prop}[definition]{Proposition}
\newtheorem{lemma}[definition]{Lemma}
\newtheorem{cor}[definition]{Corollary}
\newtheorem{ex}[definition]{Example}
\theoremstyle{remark}
\newtheorem{rem}[definition]{Remark}
\newtheorem{rem*}[definition]{}

\newcommand*\mcup{\mathbin{\mathpalette\mcupinn\relax}}
\newcommand*\mcupinn[2]{\vcenter{\hbox{$\mathsurround=0pt
  \ifx\displaystyle#1\textstyle\else#1\fi\bigcup$}}}
\newcommand*\mcap{\mathbin{\mathpalette\mcapinn\relax}}
\newcommand*\mcapinn[2]{\vcenter{\hbox{$\mathsurround=0pt
  \ifx\displaystyle#1\textstyle\else#1\fi\bigcap$}}}
\DeclarePairedDelimiter{\abs}{\lvert}{\rvert}
\DeclarePairedDelimiter{\norm}{\lVert}{\rVert}
\DeclarePairedDelimiter{\parr}{(}{)}
\DeclarePairedDelimiter{\parq}{[}{]}
\DeclarePairedDelimiter{\parqq}{\llbracket}{\rrbracket}
\DeclarePairedDelimiter{\bra}{\lbrace}{\rbrace}
\DeclarePairedDelimiter{\ceil}{\lceil}{\rceil}
\DeclarePairedDelimiter{\prodscal}{\langle}{\rangle}
\DeclarePairedDelimiter{\floor}{\lfloor}{\rfloor}
\DeclareMathOperator*{\argmin}{arg\,min}
\DeclareMathOperator*{\argmax}{arg\,max}
\DeclareMathOperator*{\expval}{\mathbb{E}}
\DeclareMathOperator*{\varval}{\mathrm{Var}}
\DeclareMathOperator*{\covval}{\mathrm{Cov}}

\begin{document}

\title{Applied Stochastic Analysis \\ Homework assignment 6}
\author{Luca Venturi}
\maketitle

\section*{Exercise 1}

\paragraph*{(a)}

First, note that if $N\sim \mathrm{Poi}(\lambda)$, then 
$$
\expval\parq{N_t} = \lambda t, \qquad \expval\parq{N_t^2} = \lambda t(1+\lambda t) \qquad \text{and} \qquad \expval\parq{X_t} = \lambda (t+\alpha) - \lambda t = \lambda \alpha.
$$
Moreover
\begin{align*}
\covval(N_t,N_s) & = \expval\parq{N_tN_s} - \expval\parq{N_t}\expval\parq{N_s} = \expval\parq{(N_{t\vee s}-N_{t\wedge s})N_{t\wedge s}} + \expval\parq{N_{t\wedge s}^2} - \expval\parq{N_t}\expval\parq{N_s} \\ & = \expval\parq{N_{t\vee s}-N_{t\wedge s}}\expval\parq{N_{t\wedge s}} + \expval\parq{N_{t\wedge s}^2} - \expval\parq{N_t}\expval\parq{N_s} = \expval\parq{N_{t\wedge s}^2} -\expval\parq{N_{t\wedge s}}^2 = \lambda ({t\wedge s}).
\end{align*}
Since $X$ is strongly stationary, its covariance function is given by
$$
C(t) = \covval(X_{\abs{t}},X_0) = \covval(N_{\abs{t}+\alpha}-N_{\abs{t}},N_\alpha) = \lambda(\alpha-\abs{t}\wedge\alpha) = \lambda(\alpha-\abs{t})\mathbbm{1}_{\bra{\abs{t}\leq\alpha}}.
$$
Since $C\in L^1(\mathbb{R})$, the spectral density exists and its given by its Fourier transform:
$$
f(\xi) = \frac{1}{2\pi}\int_\mathbb{R} \lambda(\alpha-\abs{t})\mathbbm{1}_{\bra{\abs{t}\leq\alpha}}e^{-i\xi t}dt = \frac{\lambda}{\pi} \frac{1-\cos \alpha\xi}{\xi^2}.
$$

\paragraph*{(b)}

Since $W$ is a Brownian motion, then 
$$
\expval\parq{W_t} = \expval\parq{X_t} = 0, \qquad \text{and} \qquad \expval\parq{W_tW_s} =  s\wedge t.
$$
Hence the covariance function of $X$ is given by
\begin{align*}
C(t) & = \expval\parq{X_{\abs{t}+s}X_s} = \expval\parq{(W_{\abs{t}+\alpha}-W_{\abs{t}})(W_{s+\alpha}-W_s)} \\ & = (\alpha + \abs{t}\wedge s) - \abs{t}\wedge(s+\alpha) - s\wedge(\abs{t}+\alpha) + \abs{t}\wedge s = (\alpha-\abs{t})\mathbbm{1}_{\bra{\abs{t}\leq\alpha}}
\end{align*}
As in the part (a), the spectral density exists and its given by its Fourier transform:
$$
f(\xi) = \frac{1}{2\pi}\int_\mathbb{R} (\alpha-\abs{t})\mathbbm{1}_{\bra{\abs{t}\leq\alpha}}e^{-i\xi t}dt = \frac{1-\cos \alpha\xi}{\pi\xi^2}.
$$

\section*{Exercise 2}

\paragraph*{(a)}

Using the notation on the notes, we have that the spectral representation of $X$ is 
$$
X_t = \int_\mathbb{R} e^{i\xi t}\,dZ(\xi).
$$
Derivating under the integral sign, we get the spectral representation of $X'$:
$$
X_t' = \int_\mathbb{R} e^{i\xi t}\,i\xi dZ(\xi).
$$
Its covariance function is given by
\begin{align*}
C_{X'}(t) & = \expval\parq{X_{t+s}'\overline{X_s'}} = \expval\parq*{\int_{\mathbb{R}^2} e^{i\xi (t+s)}e^{-i\xi's}\,\xi\xi'dZ(\xi)\overline{dZ(\xi')}} \\ & = \int_{\mathbb{R}^2} e^{i\xi (t+s)}e^{-i\xi's}\,\xi\xi'\expval\parq*{dZ(\xi)\overline{dZ(\xi')}} = \int_{\mathbb{R}^2} e^{i\xi (t+s)}e^{-i\xi's}\,\xi\xi'\delta(\xi-\xi')dF(\xi)d\xi' \\ & = \int_\mathbb{R} e^{i\xi t}\,\xi^2 dF(\xi) = -\frac{d^2}{dt^2}\int_\mathbb{R} e^{i\xi t}\,dF(\xi) = -\frac{d^2}{dt^2}C(t).
\end{align*}
(This confirms what found in exercise 5 of homework 5).

\paragraph*{(b)}
 
It holds that
\begin{align*}
C_{X,X'}(t) & = \expval\parq{X_{t+s}\overline{X_s'}} = -i\expval\parq*{\int_{\mathbb{R}^2} e^{i\xi (t+s)}e^{-i\xi's}\,\xi'dZ(\xi)\overline{dZ(\xi')}} \\ & = -i\int_{\mathbb{R}^2} e^{i\xi (t+s)}e^{-i\xi's}\,\xi'\expval\parq*{dZ(\xi)\overline{dZ(\xi')}} = -i\int_{\mathbb{R}^2} e^{i\xi (t+s)}e^{-i\xi's}\,\xi'\delta(\xi-\xi')dF(\xi)d\xi' \\ & = -i\int_\mathbb{R} e^{i\xi t}\,\xi dF(\xi) = -\frac{d}{dt}\int_\mathbb{R} e^{i\xi t}\,dF(\xi) = -\frac{d}{dt}C(t).
\end{align*} 

\paragraph*{(c)}

For every $t\in\mathbb{R}$ it holds
$$
\expval\parq{X_{t}\overline{X_t'}} = C_{X,X'}(0) = -C'(0).
$$
Since it is constant, its derivative is zero.

\paragraph*{(d)}

A sufficient condition is that $X$ is a $C^{2n}$ process.
 
\end{document}